\documentclass{article}
\begin{document}
	\newcommand{\HorizontalLine}{\rule{\linewidth}{0.3mm}}
	
	\title{ \HorizontalLine \\ \textbf{Interprocedual and Interpolating \\ Property Directed Reachability in \textsc{Ultimate}} \HorizontalLine}
	

	
	\author{Jonas Werner \\ Supervised By: Dr. Daniel Dietsch}


	
	\date{}
	
	\maketitle
	
	\begin{abstract}
	
	\end{abstract}
	
	
	%-------------------------------------------------%
	\section{Introduction}
	%-------------------------------------------------%
	
	In the course of my 2018 Bachelor's Thesis we devised a technique of using Property Directed Reachability on software in the software analysis framework \textsc{Ultimate}. \\
	(@ToDo\textbf{Should I to introduce Ultimate, PDR here?})
	Our approach was based on the technique described by Lange et al (@ToDo\textbf{(citation needed})).
	The implementation worked solely on intraprocedual programs, meaning that if we encountered a so called procedure call, it would no longer continue the process. \\
	Furthermore our implementation did not make use of interpolants which broaden the state space covered by blocked proof-obligations. \\
	This project aims at expanding PDR with both, interprocedual analysis capabilities and usage of interpolants. (@ToDo \textbf{Both kinds of interpolants: Craig and selfless (citation needed)}) \par
	Firstly we will give some background information about interprocedual program analysis and interpolants, then we will describe our changes made to the existing PDR library.
	
	\section{Background}
	\subsection{Interprocedual Program Analysis (in Ultimate?)}
	Describe how interprocedual program analysis works, like Internal-Call-Return Transitions, what happens between call and return, context like oldvariable nad globals, variable renaming etc etc. Define extension of CFG to \textbf{I}CFG.
	
	\subsection{Interpolation}
	What are interpolants and why are they useful? 1. Craig interpolants then selfless interpolation.
	(Maybe Pros and Cons of each?)
	
	
	\section{Interprocedual PDR in Ultimate}
	what we have done to get our intraprocedual PDR to interprocedual.
	
	\section{Using Interpolants}
	How we make use of the interpolants, meaning showing that we strengthen the frames with an interpolant to cover more predecessor states. Example maybe
	
	\section{Evaluation}
	Run the benchmarks again and compare them to the ones from last year, where we will see (hopefully) a speedup.
	
	\section{Bibliography}

	
\end{document}